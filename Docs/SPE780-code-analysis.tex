\chapter{Code Analysis}
%%%%%%%%%%%%%%%%%%%%%%%%%%%%%%%%%%%%%%%%%%%%%%%%%%%%

\section{What will be looked at}

In this chapter we will analyze the code to show the expressive nature
of the Erlang language and the ease it was to be able to create the
concurrent threading of the Brozozswki DFA generation and to make
changes for different concurrency algorithms.

The reader would be justified to keep figures \ref{fig:AsAvailable} and
\ref{fig:RoundRobin} next to the code to be able to fully understand the
code explanations. Even though we will put in an effort follow the flow of the
code, the concurrent nature of the code (as the different paths in
those figures would attest to) would not make that entirely
possible.

The most ``important'' reason the author would apologize for not
following the logical flow of the code to ease the understanding, is
the way how the functional language ordering of the code happens. Thus
even though the logical ordering/code flow is not that obvious, the
authors would argue that the code design and Erlang features would be
appreciated by understanding the reasons for the ordering as done in
this paper.

\paragraph{Chosen code to explain}

Given the bit more complexity of the nullable code of
\ref{sec:nullable}, and it being the last written code, that will be
the code set used as basis for explanation in this chapter. Certain
function names (like \texttt{hv\_brzp\_null}) should be replaced for
the previous function names used.

\paragraph{Actual Brzozoswki code}

As we have used a previously working Brozoswki sequential
implementation, and just replaced the core innerloop with our
concurrent methods, we will not discuss those code in this document as
it was outside this project's scope and function. It will have to be
stated that on analysis of those code, the author found it to be
closely follow Watson\cite{watson1995taxonomies} and that would be a good place
to research the specific workings of the Brozoswki DFA generator and
the code related to that.

\paragraph{Generated DFA correctness proofs}

The author would like to mention that it was not the scope of this work
to investigate the correctnes of the code per se. The author have,
however, checked and confirmed that the DFAs generated and used in
this work, have been verified against the original implementation,
thus errors from DFA generation, would be traced back to the original
sequential implementation used.

\section{Ab Initio}\footnote{Ab Initio: Latin for from the
  beginning. Introduced to this term from the name of a similarly
  named product/company used in massive extract-transform-load (ETL)
  environments like telcos and financials}

We will discuss the code by starting at the entry function. As this
function starts the distributor that initiates the receiver processes,
we will continue to the receiver processes. After we've explained the
distributor/receiver workings, we will jump back to the actual core
that does the decisions and work creation.


\subsection{Entry Code}

Even though it could have been parameterized more generally, the author
chose to have one function for each of the attempts/iterations of
parallelism embedded in a seperate file or module for each of the attempts.
 
However, the \textsc{RoundRobin} and \textsc{AsAvailable} schedulers have been extracted
as parameters. First lets look at the entry call for the \textsc{AsAvailable} scheduler:
\begin{lstlisting}[name=hvp2]
%The entry to hvp2
% third Parameter to chose the roundrobin or available scheduler
hv_brz(RE, Sigma, available, N) ->(*@\label{line:FuncAsav}@*)
		TimeOut = 3000,(*@\label{line:DefStarts}@*)
		Res = self(),(*@\label{self}@*)
		WiP = [],%Only the first (and last) one would be "empty"
		Finish = [],
		Dlist = [RE],%Start state.
		Delta = dict:new(),(*@\label{line:DefEnds}@*)
		Dist = spawn(fun () -> hv_dist_avail_start(TimeOut, Res, N)		 end),(*@\label{dist}@*)
		hv_brzp_null(RE, Sigma, Dist, WiP, Finish, Dlist, Delta);
\end{lstlisting}

Lines \ref{line:DefStarts}-\ref{line:DefEnds} is the parameters we used for
timeout values and the reference to ourself as the results receiver.
We initiated the work in progress (\texttt{WiP}) and expressions
already processed (\texttt{Finish}) as empty lists. \texttt{Dlist} is set to
the list containing the single element of the regular expresion to be
considered. These values are initialzed the same in all the entry functions.

Line \ref{dist} spawns the distributor process\footnote{Remember that
	Erlang processes are comparable to C/Java threads, but from the
	program's perspective they are processes}, and using the fourth
parameter (bound to the variable \texttt{N}) as the number of worker
threads to spawn. Also the timeout and our own process
information(\texttt{Res}) is passed as parameters.

We then call the actual core of the processing function (\texttt{hv\_brzp\_null}
in this case) with the parameters declared and initialized in
lines \ref{def_start}-\ref{dist}. Notice that the \texttt{Dist}ributor
from line \ref{dist} is passed as a parameter to this core processing
function, and that this line is also verbatim copy of line
\ref{processing} of the roundrobin scheduler!

Doing the \textsc{RoundRobin} scheduler, we do the same as the \textsc{AsAvailable}
scheduler, except we spawn a roundrobin distributor. However, note
that line \ref{func_rr} is the function called with the third
parameter matched as the atom \texttt{roundrobin} compared to line
\ref{func_asav} where the third parameter matched the atom
\texttt{available}.
\begin{lstlisting}[name=hvp2]
hv_brz(RE, Sigma, roundrobin, N) ->(*@\label{func_rr}@*)
(*@\textit{Repeat lines \ref{def_starts}-\ref{def_ends}}@*)
		Dist = spawn(fun () -> hv_dist_rr_start(TimeOut, Res, N) end),(*@\label{dist_rr}@*)
		hv_brzp_null(RE, Sigma, Dist, WiP, Finish, Dlist, Delta).(*@\label{processing}@*)
\end{lstlisting}

For special testing, we used a slightly modified distributor starting
function, but again we just used a function that used parameter
matching to pick the right initialization function. 
\begin{lstlisting}[name=hvp2]
%Specialized one with only two named receivers using the RoundRobin method 
hv_brz(RE,Sigma,rr_spec) ->
	(*@\textit{Repeat lines \ref{def_starts}-\ref{def_ends}}@*)
	Dist=spawn(fun() -> hv_dist_rr_spec_start(TimeOut,Res) end),
	hv_brzp_null(RE,Sigma,Dist,WiP,Finish,Dlist,Delta).
\end{lstlisting}

\subsubsection{Repeated lines}

In retrospect, we could have extracted those repeated lines and used an
intermediary function to set and pass those on, but it was not
essential to the project goals, and the gains would have been minimal
and as such not considered, though for production code this needs to
be done.


\section{Processing (Receiver)}
\label{sec:proc_receiver}

The function \texttt{hvp2:hv\_brzp\_null\/7} is the core receiver we
will discuss in this section. The reader should notice the use
of the function call pattern matching to follow the major paths in
the receiver code, and that the receiver calls itself recursively
until there is no more work in progress (WiP),
ie. tail recursion at play as explained in \ref{sec:tail-recursion}.

The difference between the empty work in progress (WiP)
\ref{sec:wip-empty} and the receive only in \ref{sec:receive}, is the
empty (\texttt{[]}) fourth parameter\ldots and it must be in that
order, else the empty WiP state would never be matched.

The derivation of a new expression is done in
\ref{sec:expr-derive}. Here (again note the order for matching the
fucntion parameters) the first parameter is the expression to derive,
and did NOT match the atom \texttt{receive\_only}!

\subsection{WiP empty (nothing to do anymore}
\label{sec:wip-empty}

\begin{lstlisting}[name=hvp2]
%The case when the WiP is empty
hv_brzp_null(receive_only,Sigma,Dist,[],Finish,Dlist,Delta) ->
	%io:format("WiP finished"), 
	Dist!{stop},
	#dfa{ states=lists:sort(Dlist), symbols=Sigma, start=lists:last(Dlist) ,transition=Delta, finals=Finish };
\end{lstlisting}
As we have already mentioned, the fourth parameter is an empty list,
and given the fact that we are only receiving (the
\texttt{receive\_only} atom matched in the first parameter) we conclude
that we are finished processing as there is no more outstanding
derivations, and we can stop the processes, thus we send the
\texttt{Dist}ributor a stop message.

Finally we will return the data in a format depicted by the
\texttt{\#dfa} record format. We will refer the author to the specifics
of this record format in \cite{joe:09}. Suficient to say that it is
nothing different from a norma tuple, just a method to name and order
the tuple into a record structure named elsewhere in the pre-compiled
code, typically a header file.

\subsection{Receive only}
\label{sec:receive}

\begin{lstlisting}[name=hvp2]
%Receive only, nothing to derive
hv_brzp_null(receive_only, Sigma, Dist, WiP, Finish, Dlist, Delta) ->
	receive
		{rd,E,I,DDI} -> (*@\label{line:rd_match}@*)
                % io:format("brzp_null_2:"),io:write({rd,E,I,DDI}),io:format("~n"),(*@\label{debug}@*)
		NewDelta=dict:store({E,I},DDI,Delta),(*@\label{line:newdelta}@*)
		case lists:member(DDI,Dlist) of(*@\label{line:case}@*)
			true ->hv_brzp_null(receive_only,Sigma,Dist, lists:delete({rd,E,I},WiP),Finish,Dlist,NewDelta);(*@\label{line:case_true}@*)
			false -> hv_brzp_null(DDI,Sigma,Dist, lists:delete({rd,E,I},WiP),Finish,[DDI|Dlist],NewDelta)(*@\label{line:case_false}@*)
		end;(*@\label{line:endcase}@*)
		{null,E,true} -> (*@\label{line:null_true}@*)
                % io:format("brzp_null_2: ~p true~n",[E]), (*@\label{debug2}@*)
			hv_brzp_null(receive_only,Sigma,Dist,
                           lists:delete({null,E},WiP),[E|Finish],Dlist,Delta);	% Add nullable states to F
		{null,E,false} ->(*@\label{line:null_false}@*)
			hv_brzp_null(receive_only,Sigma,Dist,lists:delete({null,E},WiP),Finish,Dlist,Delta)
		after 5000 ->(*@\label{line:timeout}@*)
                    io:write(WiP),
                    output_mailbox(1),
                    throw(timeoutRec_only)
	end;
\end{lstlisting}

This part of the receiver, is executed based on \texttt{receive\_only}
matching the first parameter, and it only waits for messages from the
derivation processes. This should either be a result of a derivation
tagged with the atom \texttt{rd}, or a result from a nullable test,
tagged with \texttt{null}. It would remove this request from the WiP
list, and recurse back to the same function name, but with slightly
different parameters, modified based on the result received. 

But let us have a look at the options steps a bit more closely:
\begin{description}
\item[\ref{line:rd_match}] we have received a derivation result based
  on the \texttt{rd} atom that matched.
\item[\ref{line:newdelta}] Modify the Delta dictionary by creating a
  new  dictionary with the Expression and alphabet letter
  (\texttt{{E,I}}) tuple as a key, and the derived expression
  \texttt{DDI} as the value. \textbf{Note:} the Delta variable is not
  changed at all, but a \textbf{new} variable with the name NewDelta
  is created and bounded to the new result.
\item[\ref{line:case}] Some expression and alphabet inputs might
  provide the same derivations, we check if we have seen a similar one
  before, in which case we do not waste time on deriving for the same
  path multiple times. This is done by checking for the existence of
  the DDI just received in the list Dlist, of previous expressions that
  aready have been issued derivation requests.
\item[\ref{line:case-true}] We have seen this derivation befote, so we
  just remove the {E,I} from the WiP list, and repeat the process by
  recursing back.
\item[\ref{line:case_false}] here we have not seen this derived
  expression before, thus we will call ourselves again, but this time
  with the expression that need further derivation as the first
  parameter. {E,I} is also remove from the WiP list.
\end{description}

Line \ref{line:timeout} we have assumed that after 5seconds of
inactivity, there is a problem in the distributor or processor, at
which point we exit, dumping the state of the mailbox and throwing an
exception. This was a very usefull method to track a typing mistake
where we have sent all the data, but with a mistyped atom!


\paragraph{Debugging statements}

Even though we all want to present ourselves as invincible and perfect
coders, we all make mistakes, and thus we need to debug or fault find
our code. Line \ref{debug} is a typical line used to debug the
code. This line outputs a user format string with \texttt{io:format/2} similar to the Unix
\texttt{printf(3)}\footnote{the 3 inside the brackets refers to the
  section 3 manual pages, which is the ``standard'' libc functions
  on Unix}. This is  then followed by the Erlang formating of the
data structure passed to \texttt{io:write/1}, and finally a
linefeed. Line \ref{debug2} shows an example with some substitution
where the \texttt{~p} is substitute with the value bound to the
variable \texttt{E} and note that it is a list \texttt{[E]} passed on
to the function, and the substitution happens based on the list positions.

\subsubsection{}

\subsection{Expression to derive}
\label{sec:expr-derive}

This part of the function, have the same code than the receive\_only in
lines\ref{line:rec2}-\ref{line:rec2end}. It is lines
\ref{line:E2}-\ref{line:E2e} that is the interesting part in this
function part.

\begin{lstlisting}[name=hvp2]
% When we have an RE/E d/di that needs to be derived/etc.
hv_brzp_null(E,Sigma,Dist,WiP,Finish,Dlist,Delta) ->
	%foreach(Sigma) send message to Dist
	lists:foreach(fun(X) -> Dist!{process,[rd,E,X]} end,Sigma),(*@\label{line:E2}@*)
	%F1=nullable(E,Finish),nullable(RE),(*@\label{line:null}@*)
	Dist!{process,[null,E]},(*@\label{line:dnull}@*)
	%foreach(Sigma) insert {E,I} into WiP, and add the null to the begining ;)
	NewWiP=[{null,E}|add_wip(WiP,rd,E,Sigma)],(*@\label{line:E2e}@*)
	
	%WiP would not be empty in this function :)
	receive (*@\label{line:rec2}@*)
		{rd,E,I,DDI} -> %io:format("brzp_null_why:	"),io:write({rd,E,I,DDI}),io:format("~n"),%"~p ~p ~p~n",[E,I,DDI]),
			NewDelta=dict:store({E,I},DDI,Delta),
			case lists:member(DDI,Dlist)	of
				true -> hv_brzp_null(receive_only,Sigma,Dist,lists:delete({rd,E,I},NewWiP),Finish,Dlist,NewDelta);
				false -> hv_brzp_null(DDI,Sigma,Dist,lists:delete({rd,E,I},NewWiP),Finish,[DDI|Dlist],NewDelta)
			end;
		{null,E,true} -> % io:format("brzp_null: ~p true~n",[E]),
							hv_brzp_null(receive_only,Sigma,Dist,lists:delete({null,E},NewWiP),[E|Finish],Dlist,Delta);	% Add nullable states to Finish
		{null,E,false} -> %io:format("brzp_null: ~p false~n",[E]),
			hv_brzp_null(receive_only,Sigma,Dist,lists:delete({null,E},NewWiP),Finish,Dlist,Delta)
		%Other -> throw (Other)
		after 5000 ->	io:write(WiP),throw(timeOut)
end.(*@\label{line:rec2end}@*)

\end{lstlisting}

\begin{description}
  \item[\ref{line:E2}] This is an example where we apply a function
    over the Sigma alphabet. The ``function'' in this case, is to sent
    a message to the distributor \texttt{Dist}. This message consists
    of the tuple \texttt{{process,[rd,E,X]}}. For some reason I made
    it a two pair tuple, with the parameters a list, rather than an
    extended tuple.
  \item[\ref{line:null}] The non-distributing versions, used theses
    lines to check for the end conditions and whether the expression
    is 
\end{description}


\section{Receivers}

\subsection{Round Robin}
\begin{lstlisting}[name=hvp2]
%The receiver for the round robin/sequential tests
%
hv_rr_rec(Name,Res) -> 
	%io:format(" starting: ~p self:~p Res:~p~n",[Name,pid_to_list(self()),pid_to_list(Res)]),
	receive 
		{stop} -> %io:format("stopping ~p ~n",[Name]),
					false;
		{process,[rd,E,I]} -> Res!{rd,E,I,mds:reduce(mds:deriv(E,I))},
						 hv_rr_rec(Name,Res);
		{process,[null,E]} -> Res!{null,E,mds:null(E)},
													 hv_rr_rec(Name,Res);
		Other -> io:write(Other),throw(Other)
		after 3000 ->	io:format("Timeout ~p quiting",[Name]),io:nl()
	end.
\end{lstlisting}

\subsection{AsAvailable Receiver}
\begin{lstlisting}[name=hvp2]

%===============================================
%The distributor that	 have receivers that tells it when 
% they are finished and ready for new processing
%===============================================

%Let the Distributor know when finished with processing
% But this state engine not rock solid when more than one messages was sent 
% etc.
hv_rec_available(Timeout,Name,Res,Dist) ->
	%io:format("~nEntering ~p ~p ~p ~p ~n",[Name,integer_to_list(Timeout),pid_to_list(Res),pid_to_list(Dist)]),
	%First we handle all stop/process messages on the queue
	%If the distributor works correct, this shouldn't be necesssary,
	%But we could "modify" the distributor to send expected "short"
	% requests in rapid fire??
	receive
		{stop} ->% io:format("stopping ~p~n",[Name]),
					exit(0); %Need to do the exit here else continue to next
		{process,[rd,E1,I1]}-> %io:format("~p ~p ~p~n",[Name,E1,I1]),
			Res!{rd,E1,I1,mds:reduce(mds:deriv(E1,I1))},
			hv_rec_available(Timeout,Name,Res,Dist);
		{process,[null,E1]} -> %io:format("Null: ~p ~p ~n",[Name,E1]),
			Res!{null,E1,mds:null(E1)},
			hv_rec_available(Timeout,Name,Res,Dist);
		Other1 -> throw(Other1)
		after 0 -> Dist!{available,self()} %Nothing in queue, so we let the Distributor know
	end,
	receive %There were no "normal"/"expected" messages in the queue, so lets wait :)
		{stop} -> %io:format("stopping ~p~n",[Name]),
					true;
		{process,[rd,E,I]} -> %io:format("~p ~p ~p~n",[Name,E,I]),
			Res!{rd,E,I,mds:reduce(mds:deriv(E,I))},
			hv_rec_available(Timeout,Name,Res,Dist);
		{process,[null,E]} -> %io:format("Null: ~p ~p ~n",[Name,E]),
			Res!{null,E,mds:null(E)},
			hv_rec_available(Timeout,Name,Res,Dist);
		Other -> throw(Other)
		after Timeout ->
			io:format("Timeout ~p quiting ~n",[Name])
	end.

\end{lstlisting}

\begin{lstlisting}[name=hvp2]
%Start N round-robin receivers that will send their results to Res
% returning the list of	 PIDs.
list_start_servers(0,_Res) -> [];
list_start_servers(N,Res) -> 
	[spawn(fun()->hv_rr_rec("Receiver "++[N+\$0],Res)
				end)|list_start_servers(N-1,Res)].
\end{lstlisting}%$

\begin{lstlisting}[name=hvp2]
%Number of servers variable, should make that a number to pass too, but 
% for the moment this is adequate to test etc.
hv_dist_rr_start(TimeOut,Res,N) -> 
	hv_dist_rr(list_start_servers(N,Res),TimeOut).

%Two specified servers
hv_dist_rr_spec_start(TimeOut,Res)->
	Rec1=spawn(fun() ->hv_rr_rec("Rec1",Res) end),receive after 100 -> true end,
	Rec2=spawn(fun() ->hv_rr_rec("Rec2",Res) end),receive after 100 -> true end,
	hv_dist_rr([Rec1,Rec2],TimeOut).
\end{lstlisting}

\subsection{Round Robin distributor}
\begin{lstlisting}[name=hvp2]
%Round Robin distributor.. we know this is not "optimal" :)
hv_dist_rr([H|T]=Receivers,TimeOut) ->
	%io:format("Dist_rr starting: SendTo: ~p Self:~p ~n",[pid_to_list(H),pid_to_list(self())]),
	receive
		{stop} -> lists:foreach(fun(X)->X!{stop} end,Receivers);
		{process,Param} -> H!{process,Param},hv_dist_rr(lists:append(T, [H]),TimeOut);
		Other -> io:write(Other),throw(Other)
		after TimeOut ->
			io:format( "Dist quiting and stopping receivers"),
			lists:foreach(fun(X)->X!{stop} end, Receivers)
	end.
\end{lstlisting}

\section{AsAvailable distributor}
\begin{lstlisting}[name=hvp2]
%The Available distributor
%First the "empty" case
hv_dist_available(Timeout,[]) ->
	%io:format("Entering dist_available []~n"),
	receive
		{available,PID}->
			receive %We check for the availability of a process message to "fast track", else call the normal wait
				{process,Param}->PID!{process,Param},hv_dist_available(Timeout,[]) %Goody! a message available
				after 0-> hv_dist_available(Timeout,[PID]) %Normal wait since no process message in mailbox
			end;
		{stop} -> io:format("Distributor stopping from empty state~n") %no available receivers to stop :(
		after Timeout ->
			io:format("timeout distributor from waiting state~n")
	end;
hv_dist_available(Timeout,[H|Tail])-> %At least have a PID to check
	%io:format("Entering dist_available with~n"),
	receive
		{available,PID}->hv_dist_available(Timeout,[H,PID|Tail]); %H or PID, shouldn't matter which is first'
		{process,Param}->H!{process,Param},
						 hv_dist_available(Timeout,Tail);
		{stop}-> lists:foreach(fun(X)->X!{stop} end,[H|Tail]); %Stop all the available receivers
		Other -> throw(Other)
		after Timeout ->
			io:format("Timeout distributor fron available state~n")
	end.

\end{lstlisting}

\begin{lstlisting}[name=hvp2]
%Start the receivers and the distributor
hv_dist_avail_start(Timeout,_Res,0)	->
		io:format("Empty~n"),
		hv_dist_available(Timeout,[]);
hv_dist_avail_start(Timeout,Res,N) when N>0 ->
	%io:format("~n~p: Dist avail ~p ~p ~p~n~n",
	%			[pid_to_list(self()),erlang:integer_to_list(Timeout),
	%			 erlang:pid_to_list(Res),erlang:integer_to_list(N)]),
	Dist=self(),
	spawn(fun() -> hv_rec_available(Timeout,"Receiver "++erlang:integer_to_list(N),Res,Dist) end),
	hv_dist_avail_start(Timeout,Res,N-1).
\end{lstlisting}

\section{Anxillary functions}
\subsection{Adding to Work in Progress}

The Work in Progres (WiP) is a list of the type of processing that
have been sent out to the distributors, and that we still have not
received any responses back for. It is a (in our implementation) a
simple list, but is perhaps a place to optimize if it grows too
big. This is especially as we prepend the new work to the beginning of
the list, but the expected responses to be received next are those at
the back of the list that would need to be removed, so in retrospect a
FIFO type queue would be a prefered implementation when optimizing.

\begin{lstlisting}[name=hvp2]
%For all the Sigma add {E,i} to the Work In Progress
add_wip(WiP, Type, E, [H]) -> [{Type, E, H}| WiP];(*@\label{AddHead}@*)
add_wip(WiP, Type, E, [H| SigmaT]) -> add_wip([{Type, E, H}| WiP], Type, E, SigmaT).(*@\label{line:AddList}@*)
\end{lstlisting}

Again we saw the use of specialization by the use of parameter
matching. Here we first check on line~\ref{line:AddHead} if the fourth
parameter is a list with a single element (and we bound the variable
\texttt{H} to that single element), else we match on
line~\ref{line:AddList} for a list with more than one element and we bind
the head (first element) to \texttt{H} while the tail (rest of the list
excluding the head) is bound to \texttt{SigmaT}.

\subsubsection{Mailbox debugging}

\begin{lstlisting}[name=hvp2]
 output_mailbox(N) ->
	receive
		Mess -> io:format("Message ~p~n:",[N]),
								io:write(Mess), io:nl(),
								output_mailbox(N+1)
	after 0 -> exit(123)
	end.
\end{lstlisting}



