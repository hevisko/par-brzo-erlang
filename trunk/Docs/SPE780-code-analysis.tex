\chapter{Code Analysis}
%%%%%%%%%%%%%%%%%%%%%%%%%%%%%%%%%%%%%%%%%%%%%%%%%%%%

\section{General code remarks}


After spending about a weekend coding the second and third iterations,
the author have been impressed by the expressiveness of the Erlang
language and the ease of the concurrency in this project. It has to be
said that once you understand the pattern matching principles of
Erlang, the coding do get easier and much more expressive than similar
code the author had to write in C/C++.

The author do apologize for the less than profesional looking code
formatting, as the ability to output properly formatted code, proofed
to be very much the same research effort than the actual code written in Erlang,
(which was not the research focus of this project).

\section{What will be looked at}

In the previous chapters we described the history and gave some
overviews of Erlang and the algorithms used.  In this chapter we will
analyze the code to show the expressive nature of the Erlang language
and the ease it was to be able to create the concurrent threading of
the Brzozowski DFA generation.

The reader would be justified to keep \autoref{fig:AsAvailable},
\autoref{fig:distflow} and \autoref{fig:distflownull} next to the code
to be able to fully understand the code explanations. Even though we
will put in an effort follow the flow of the code, the concurrent
nature of the code (as the different paths in those figures would
attest to) would not make that entirely possible.

The most ``important'' reason the author would apologize for not
following the logical flow of the code to ease the understanding, is
the way how the functional language ordering of the code happens. Thus
even though the logical ordering/code flow is not that obvious, the
authors would argue that the code design and Erlang features would be
appreciated by understanding the reasons for the ordering as done in
this paper.

\subsection{Chosen code to explain}

Given the bit more complexity of the nullable code of
\autoref{sec:nullable}, and it being the last written code, the author
decided to use that code as basis for explanation in this chapter.

\subsection[\emph{Ab Initio}]{\emph{Ab Initio}\footnote{\emph{Ab
      Initio}: Latin for \emph{from the beginning}. Introduced to this
    term from the name of a product/company used in massive
    extract-transform-load (ETL) environments like telcos and
    financials}}

Referring to \autoref{fig:mapreduce}, we will start to discuss the
code by starting at the request and entry function.
This being the begining of the code flow, as this would set the parameters
for the rest of the code portions discussed. This process PID is the
results receiver, typically designated with the \texttt{Res}
variable. This portion also spawns the distributor, and bounds the PID
to the \texttt{Dist} variable that gets passed on to the core result
aggregator function to use as destination for the processing requests.

From there we will look at the Result Aggregator in
\autoref{sec:proc_receiver}, which is the core of the processing. This
function will generate the requests, and pass them on to the
\texttt{Dist}ributor, while itself is the \texttt{Res}ults receiver
that reduces the mapped data.

After that we will discuss the two different mappers we have
implemented. They (having their entry point the \texttt{Dist}ributor)
send their processed output to the \texttt{Res}ult receiver.

Lastly we will tied up some loose ends in discussing some of the
ancillary code portions used to keep the code clean and simple.

\textbf{Note:} this is not the actual chronological execution of the
code, through time and code flow, but rather the flow of the
Brzozowski algorithm's processing.

\subsection{Actual Brzozowski code}

As we have used a previously working Brozoswki sequential
implementation, and just replaced the core inner loop with our
concurrent methods, we will not discuss that part of the code in this
document as it was outside this project's scope and function. It will
be stated that a brief overview analysis of that code, the author
found it to be closely following Watson\cite{watson1995taxonomies} and
that would be a good place to research the specific workings of the
Brozoswki DFA generator and the code related to that.

\subsection{Generated DFA correctness proofs}

The author would like to mention that it was not the scope of this
work to investigate the correctness of the code per se. The author
have, however, checked and confirmed that the DFAs generated and used
in this work, have been verified against the original implementation,
thus errors from DFA generation, would be traced back to the original
sequential implementation used.

\subsection{Code formatting}
\label{sec:code-formatting}

Typesetting is never the easiest of things, and typically needs to be
taken into consideration as the code is written. This is not the case
when the code is first developed, and then we try to type set the
code. Though there have been some effort in formatting the code in a
readable format, this was not always achieved, especially given the
limited width in typesetting, while the actual code is viewed and
developed on wide screen displays.

To help with keeping the references as close to the original code as
possible, we have used an auto wrapping features of the \LaTeX\ style
package called \texttt{\textbf{listings}}. And to indicate where the
source code's original lines are, we have numbered them. Thus those
lines without a number in front of them, is still continuations of the
numbered line above them.

\paragraph{Comments}
\label{sec:comments}

We have to point out that the $\%$ sign is used to indicate the start
of a comment line, and those have also been typeset in \emph{italics}.

\section{Entry Code}

Even though it could have been parameterized more generally, the author
chose to have one function for each of the different implementations,
and also put it in a separate module.

However, the \textsc{RoundRobin} and \textsc{AsAvailable} schedulers have been extracted
as parameters. First let us look at the entry call for the \textsc{AsAvailable} scheduler:
\begin{lstlisting}[name=hvp2]
%The entry to hvp2
% third Parameter to chose the roundrobin or available scheduler
hv_brz(RE, Sigma, available, N) ->(*@\label{line:FuncAsav}@*)
		TimeOut = 3000,(*@\label{line:DefStarts}@*)
		Res = self(),(*@\label{line:self}@*)
		WiP = [],%Only the first (and last) one would be "empty"
		Finish = [],
		Dlist = [RE],%Start state.
		Delta = dict:new(),(*@\label{line:DefEnds}@*)
		Dist = spawn(fun () -> hv_dist_avail_start(TimeOut, Res, N)		 end),(*@\label{line:dist}@*)
		hv_brzp_null(RE, Sigma, Dist, WiP, Finish, Dlist, Delta); (*@\label{line:corecall}@*)
\end{lstlisting}

\begin{description}
\item[\autoref{line:DefStarts}-\ref{line:DefEnds}] is the parameters
  we used for timeout values and the reference to ourself (see
  \autoref{sec:PID}) as the results receiver.  We initiated the work
  in progress (\texttt{WiP}) and expressions already processed
  (\texttt{Finish}) as empty lists. \texttt{Dlist} is set to the list
  containing the single element of the regular expresion to be
  considered. These values are initialized the same in all the entry
  functions.

\item[\autoref{line:dist}] spawns the distributor
  process\footnote{Remember that Erlang processes are comparable to
    C/Java threads, but from the program's perspective they are
    processes}, and using the fourth parameter (bound to the variable
  \texttt{N}) as the number of worker threads to spawn. Also the
  timeout and our own process information(\texttt{Res}) is passed as
  parameters.

  \textbf{Note:} The distributor could also be a single process, not
  doing any extra threading/distribution (and in retrospect should
  have been the special serialized test case), as the actual
  reducer/inner core do not care about the way the processing is
  spread, just that it sent it to this process, and then receive
  responses back. For that reason we will first look at the core
  reducer/result aggregator, and lastly at the distributors and
  receivers.

\item[\autoref{line:corecall}] Lastly the call to the actual core of
  the processing function (\texttt{hv\_brzp\_null} in this case) with
  the parameters declared and initialized in \autoref{line:DefStarts}
  to \autoref{line:DefEnds}. Notice that the \texttt{Dist}ributor from
  \autoref{line:dist} is passed as a parameter to this core processing
  function, and that this line is also verbatim copy of line
  \autoref{line:processing} of the round robin scheduler!
\end{description}

Doing the \textsc{RoundRobin} scheduler, we do the same as the
\textsc{AsAvailable} scheduler, except we spawn a round robin
distributor. However, note that line \autoref{line:func_rr} is the function
called with the third parameter matched as the atom
\texttt{roundrobin} compared to line \autoref{line:func_asav} where the
third parameter matched the atom \texttt{available}.
\begin{lstlisting}[name=hvp2]
hv_brz(RE, Sigma, roundrobin, N) ->(*@\label{line:func_rr}@*)
(*@\textit{Repeat lines \autoref{def_starts}-\autoref{def_ends}}@*)
		Dist = spawn(fun () -> hv_dist_rr_start(TimeOut, Res, N) end),(*@\label{line:dist_rr}@*)
		hv_brzp_null(RE, Sigma, Dist, WiP, Finish, Dlist, Delta).(*@\label{line:processing}@*)
\end{lstlisting}

For special testing, we used a slightly modified distributor starting
function, but again we just used a function that used parameter
matching to pick the right initialization function. 
\begin{lstlisting}[name=hvp2]
%Specialized one with only two named receivers using the RoundRobin method 
hv_brz(RE,Sigma,rr_spec) ->
	(*@\textit{Repeat lines \autoref{def_starts}-\autoref{def_ends}}@*)
	Dist=spawn(fun() -> hv_dist_rr_spec_start(TimeOut,Res) end),
	hv_brzp_null(RE,Sigma,Dist,WiP,Finish,Dlist,Delta).
\end{lstlisting}



\subsection{Repeated lines}

In retrospect, we could have extracted those repeated lines and used an
intermediary function to set and pass those on, but it was not
essential to the project goals, and the gains would have been minimal
and as such not considered, though for production code this needs to
be done.



\section{Result Aggregator a.k.a. Reducer}
\label{sec:proc_receiver}

The function \texttt{hvp2:hv\_brzp\_null\/7} is the core of the whole
algorithm and it implements \autoref{fig:distflownull}. We have
explained in \autoref{fig:mapreduce} how this function relates to the
reducer in the Google MapReduce.

 The reader should notice the use of the
function call pattern matching to follow the major paths in the
receiver code, and that the receiver calls itself recursively until
there is no more work in progress (WiP), ie. tail recursion at play as
explained in \autoref{sec:tailrec}.

The difference between the empty work in progress (WiP)
\autoref{sec:wip-empty} and the receive only in \autoref{sec:receive},
is the empty (\texttt{[]}) fourth parameter\ldots and it must be in
that order, else the empty WiP state would never be matched.

The derivation of a new expression is done in
\autoref{sec:expr-derive}. Here (again note the order for matching the
function parameters) the first parameter is the expression to derive,
and did NOT match the atom \texttt{receive\_only}!

\subsection{WiP empty (nothing to do anymore}
\label{sec:wip-empty}

\begin{lstlisting}[name=hvp2]
%The case when the WiP is empty
hv_brzp_null(receive_only,Sigma,Dist,[],Finish,Dlist,Delta) ->
	%io:format("WiP finished"), 
	Dist!{stop},
	#dfa{ states=lists:sort(Dlist), symbols=Sigma, start=lists:last(Dlist) ,transition=Delta, finals=Finish };
\end{lstlisting}
As we have already mentioned, the fourth parameter is an empty list,
and given the fact that we are only receiving (the
\texttt{receive\_only} atom matched in the first parameter) we conclude
that we are finished processing as there is no more outstanding
derivations, and we can stop the processes, thus we send the
\texttt{Dist}ributor a stop message.

Finally we will return the data in a format depicted by the
\texttt{\#dfa} record format. We will refer the author to the specifics
of this record format in \cite{joe:09}. Sufficient to say that it is
nothing different from a norma tuple, just a method to name and order
the tuple into a record structure named elsewhere in the pre-compiled
code, typically a header file.

\subsection{Receive only}
\label{sec:receive}

This part of the receiver, is executed based on \texttt{receive\_only}
matching the first parameter, and it only waits for messages from the
derivation processes. This should either be a result of a derivation
tagged with the atom \texttt{rd}, or a result from a nullable test,
tagged with \texttt{null}. It would remove this request from the WiP
list, and recurse back to the same function name, but with the
parameters modified based on the result received.
\begin{lstlisting}[name=hvp2]
%Receive only, nothing to derive
hv_brzp_null(receive_only, Sigma, Dist, WiP, Finish, Dlist, Delta) ->(*@\label{line:rec_only}@*)
	receive
		{rd,E,I,DDI} -> (*@\label{line:rd_match}@*)
		%io:format("brzp_null_2:"), io:write({rd,E,I,DDI}), io:format("~n"),(*@\label{debug}@*)
		NewDelta=dict:store({E,I},DDI,Delta),(*@\label{line:newdelta}@*)
		case lists:member(DDI,Dlist) of(*@\label{line:case}@*)
			true ->hv_brzp_null(receive_only,Sigma,Dist, lists:delete({rd,E,I},WiP),Finish,Dlist,NewDelta);(*@\label{line:case_true}@*)
			false -> hv_brzp_null(DDI,Sigma,Dist, lists:delete({rd,E,I},WiP),Finish,[DDI|Dlist],NewDelta)(*@\label{line:case_false}@*)
		end;(*@\label{line:endcase}@*)
		{null,E,true} -> (*@\label{line:null_true}@*)
			%io:format("brzp_null_2: ~p true~n",[E]), (*@\label{line:debug2}@*)
			hv_brzp_null(receive_only, Sigma, Dist, lists:delete({null,E},WiP), [E|Finish], Dlist, Delta);	% Add nullable states to F
		{null,E,false} ->(*@\label{line:null_false}@*)
			hv_brzp_null(receive_only, Sigma, Dist, lists:delete({null,E},WiP), Finish, Dlist, Delta)
		after 5000 ->(*@\label{line:timeout}@*)
					io:write(WiP),
					output_mailbox(1),
					throw(timeoutRec_only)
	end;
\end{lstlisting}
But let us have a look at the options steps a bit more closely:
\begin{description}
\item[\autoref{line:rec_only}] This function matched the atom \texttt{receive\_only} in the first parameter. When there is an expression to derive, it would not match this function header, but rather \autoref{line:E2} in \autoref{sec:expr-derive}
\item[\autoref{line:rd_match}] we have received a derivation result based
  on the \texttt{rd} atom that matched.
\item[\autoref{line:newdelta}] Modify the Delta dictionary by creating a
  new  dictionary with the Expression and alphabet letter
  (\texttt{{E,I}}) tuple as a key, and the derived expression
  \texttt{DDI} as the value.

 \textbf{Note:} the Delta variable is not
  changed at all, but a \textbf{new} variable with the name NewDelta
  is created and bounded to the new result.
\item[\autoref{line:case}] Some expression and alphabet inputs might
  provide the same derivations, we check if we have seen a similar one
  before, in which case we do not waste time on deriving for the same
  path multiple times. This is done by checking for the existence of
  the DDI just received in the list Dlist, of previous expressions that
  already have been issued derivation requests.
\item[\autoref{line:case_true}] We have seen this derivation before, so we
  just remove the {E,I} from the WiP list, and repeat the process by
  recursing back.
\item[\autoref{line:case_false}] here we have not seen this derived
  expression before, thus we will call ourselves again, but this time
  with the expression that need further derivation as the first
  parameter. {E,I} is also remove from the WiP list.
\end{description}

Line \autoref{line:timeout} we have assumed that after 5seconds of
inactivity, there is a problem in the distributor or processor, at
which point we exit, dumping the state of the mailbox and throwing an
exception. This was a very useful method to track a typing mistake
where we have sent all the data, but with a mistyped atom!


\paragraph{Debugging statements}

Even though we all want to present ourselves as invincible and perfect
coders, we all make mistakes, and thus we need to debug or fault find
our code. Line \autoref{debug} is a typical line used to debug the
code. This line outputs a user format string with \texttt{io:format/2} similar to the Unix
\texttt{printf(3)}\footnote{the 3 inside the brackets refers to the
  section 3 manual pages, which is the ``standard'' libc functions
  on Unix}. This is  then followed by the Erlang formatting of the
data structure passed to \texttt{io:write/1}, and finally a
linefeed. Line \autoref{debug2} shows an example with some substitution
where the \texttt{~p} is substitute with the value bound to the
variable \texttt{E} and note that it is a list \texttt{[E]} passed on
to the function, and the substitution happens based on the list positions.


\subsection{Expression to derive}
\label{sec:expr-derive}

This part of the function, have the same code than the receive\_only in
\autoref{line:rec2}-\ref{line:rec2end} (except the debugging and
comments removed). It is 
\autoref{line:E2} to \autoref{line:NewWip} that is the interesting part in this
function.

\begin{lstlisting}[name=hvp2]
% When we have an RE/E d/di that needs to be derived/etc.
hv_brzp_null(E,Sigma,Dist,WiP,Finish,Dlist,Delta) ->(*@\label{line:E2}@*)
	%foreach(Sigma) send message to Dist:
	lists:foreach(fun(X) -> Dist!{process,[rd,E,X]} end,Sigma),(*@\label{line:distribute}@*)
	Dist!{process,[null,E]},(*@\label{line:dnull}@*) %And then also sent a nullable process request
	NewWiP=[{null,E}|add_wip(WiP,rd,E,Sigma)],(*@\label{line:NewWip}@*) % foreach(Sigma) insert {E,I} into WiP, and add the null to the beginning ;)
\end{lstlisting}

\begin{description}
\item[\autoref{line:E2}] If the function parameter matching have not
  matched the atom \texttt{receive\_only} in \autoref{line:rec_only},
  we consider this to be an expression that needs to be further
  derived, and the variable \texttt{E} gets bound to that expression.
\item[\autoref{line:distribute}] This line generates
  the messages for the derivations that need to be processed. The function
  \lstinline|fun(X) -> Dist!{process,[rd,E,X]} end|, gets applied
  \texttt{foreach/2} element \texttt{X} in the list \texttt{Sigma}, and sends out the tuple
  \texttt{\{process,[rd,E,X]\}} to the \texttt{Dist}ributor, where:
  \begin{description}
  \item[\texttt{rd}] is the atom to indicate that we need to do a
    derivation. Compare that to \autoref{line:dnul}, where the
    \texttt{null} would indicate that a nullable check should be done.
  \item[\texttt{E}] The expression that was passed on to this
    function, and that needs to be further derived.
  \item[\texttt{X}] The element that we need to derive on,
    ie. $\frac{d}{dX}E$ where $X\in \Sigma$
  \end{description}
 Thus, with $X\in \Sigma$, and the atom
  \texttt{rd} indicating that the request is to process and return $\frac{d}{dX}E$

  A choice during the implementation was made to use a two pair tuple,
  with the parameters a list, rather than an extended tuple. This
  choice was just for convenience, but further implementations should
  consider making it rather an extended tuple.
\item[\autoref{line:dnull}] sends the \texttt{Dist}ributor a request to
  process a nullable test of the expression \texttt{E}
\item[\autoref{line:NewWip}] generates the new work in progress
  list. This is done in two steps. First it calls \texttt{add\_wip/4}
  (\autoref{sec:adding-work-progress}), that extends the current
  \texttt{WiP} with the same list of \texttt{\{rd,E,X\}} that have
  been sent to the \texttt{Dist}ributor. Then, using the
  \texttt{[Head|Tail]} list expansion method, prepends the
  \texttt{\{null,E\}} request to the list bound to \texttt{NewWip}
\end{description}

The rest of this function (\autoref{line:rec2} to
\autoref{line:rec2end}), is the same as that in \autoref{sec:receive},
with the exception of using the newly generated \texttt{NewWip}
instead \texttt{WiP} passed to the function
\begin{lstlisting}[name=hvp2]
	receive (*@\label{line:rec2}@*)
		{rd,E,I,DDI} -> 
			NewDelta=dict:store({E,I},DDI,Delta),
			case lists:member(DDI,Dlist)	of
				true -> hv_brzp_null(receive_only, Sigma, Dist, lists:delete({rd,E,I},NewWiP), Finish, Dlist, NewDelta);
				false -> hv_brzp_null(DDI, Sigma, Dist, lists:delete({rd,E,I}, NewWiP), Finish, [DDI|Dlist], NewDelta)
			end;
		{null,E,true} -> hv_brzp_null(receive_only, Sigma, Dist, lists:delete({null,E},NewWiP), [E|Finish], Dlist, Delta);	
		{null,E,false} ->	hv_brzp_null(receive_only, Sigma, Dist, lists:delete({null,E},NewWiP), Finish, Dlist, Delta)
		after 5000 ->	io:write(WiP),throw(timeOut)
end.(*@\label{line:rec2end}@*)

\end{lstlisting}


\section{Mappers}

Initially (and in the code) referred to as the distributor
and receiver pairs, these pairs are the Google MapReduce mappers, as explained and shown
in \autoref{sec:map-Reduce-google} and \autoref{fig:mapreduce}.

The reader should refer to \autoref{sec:distribution-queues} for
algorithmic choices and explanations. What we can expand on in this
section, is to mention that the threads are started by the
distributor when the distributor is initialized. The distributor also
sends the stop messages to the receiver threads.

The difference between the two type of mappers, is the way the
distributor keeps track of the receivers and whether the receivers
give any feedback to the distributor or not. With that in mind, we
will highlight the differences when we discuss the AsAvailable mapper
pair, but explain the main interactions and flows with the RoundRobin mappers.

\subsection{Round Robin mappers}

The Round Robin receiver is very simple, in so far as it just process
a matched message, sends on the result, and then executes itself recursively\ldots until it
receives a \texttt{\{stop\}} message.
The RoundRobin distributor is a bit more involved, as it needs to
rotate the list of receivers.


\subsubsection{RoundRobin Distributor}
\label{sec:roundr-distr}

The distributors consists of two part, the first being the entry
portion that spawns the threads,
and the actual distributor. 

\paragraph{Spawning the receivers}
\label{sec:spawning-receivers}

After the parameters for the algorithm (like the number of threads to
use etc.) have been configured, we create a number of threads based on
the parameters passed on to the distributor as can be seen on
\autoref{line:rr_d_start}, that then call the real RoundRobin
distributor with a list of receivers on \autoref{line:rr-d-call}

\begin{lstlisting}[name=hvp2]
%Number of servers variable, should make that a number to pass too, but 
% for the moment this is adequate to test etc.
hv_dist_rr_start(TimeOut,Res,N) -> (*@\label{line:rr_d_start}@*)
	hv_dist_rr(list_start_servers(N,Res),TimeOut). (*@\label{line:rr-d-call}@*)
\end{lstlisting}

The \texttt{list\_start\_servers/2} function recursively (because
Erlang, being a functional language, do not have an iterative for-next
loop), generates a list of PIDs, by prepending the list with the
result of the \texttt{spawn/1} function on \autoref{line:lss-head},
that have started a thread running the \texttt{hv\_rr\_rec} function,
described in \autoref{sec:roundrobin-receiver}.

\begin{lstlisting}[name=hvp2]
%Start N round-robin receivers that will send their results to Res
% returning the list of	 PIDs.
list_start_servers(0,_Res) -> []; (*@\label{line:lss0}@*)
list_start_servers(N,Res) -> (*@\label{line:lss}@*)
	[spawn(fun()->hv_rr_rec("Receiver "++[N+\$0],Res), end) (*@\label{lss-head}@*)
   |list_start_servers(N-1,Res)].(*@\label{lss-tail}@*)
\end{lstlisting}

The simplified special startup case I initial used to test the
receiver algorithm's workings is listed in \autoref{line:spec_start}
to \autoref{line:spec_end}. The \texttt{list\_start\_servers/2} with a
parameter of \texttt{N=2} would have the same effect, just here the
reader can clearly see the functional workings of the spawning of the
receivers!


\begin{lstlisting}[name=hvp2]
%Two specific and specified servers
hv_dist_rr_spec_start(TimeOut,Res)->(*@\label{line:spec_start}@*)
	Rec1=spawn(fun() ->hv_rr_rec("Rec1",Res) end) end,
	Rec2=spawn(fun() ->hv_rr_rec("Rec2",Res) end) end,
	hv_dist_rr([Rec1,Rec2],TimeOut).(*@\label{line:spec_end}@*)
\end{lstlisting}
 

\paragraph{Chain of processes}
\label{sec:chain-processes}

Note how we received the PID of the results receiver from the output
of \texttt{self/1} in the initialization function, that passed it on
to the distributor via the \texttt{spawn/1} function's parameter
function. That \texttt{spawn/1}'s output PID is then saved and used as
the next destination for sending the processing requests.

This process is then extended here, where we pass the \texttt{Res} PID
yet again to the receivers when we initialized them (Note, the Round
Robin receivers never know their parent!). In the Round Robin
distributor, we \textbf{do} save the output of the \texttt{spawn/1},
because we will use that to decide who is next inline.



\paragraph{The Real Distributor}
Now that all the receivers have been started (but not yet discussed
which follow in \autoref{sec:roundrobin-receiver}), the real
distributor function on \autoref{line:real_dist_start} gets called
with a list of receivers and the \texttt{TimeOut} value.

\begin{lstlisting}[name=hvp2]
hv_dist_rr([H|T]=Receivers,TimeOut) ->(*@\label{line:real_dist_start}@*)
	receive
		{stop} -> lists:foreach(fun(X)->X!{stop} end,Receivers);
		{process,Param} -> 
			H!{process,Param}, (*@\label{line:send_rec}@*)
			hv_dist_rr(lists:append(T, [H]),TimeOut);(*@\label{line:dist_cont}@*)
		Other -> io:write(Other),throw(Other)
		after TimeOut ->
			io:format( "Dist time-out and stopping receivers"),
			lists:foreach(fun(X)->X!{stop} end, Receivers)
	end.
\end{lstlisting}

Looking again at that, function while documenting the code, the author had to
pause yet again, and appreciate the beauty and elegance of the
functional language aspect of Erlang.

Given the details of code explanations given thus far, the author
would only highlight the \texttt{[H|T]=Receivers} construct and
point the reader to the \texttt{list:append(T,[H])} on
\autoref{line:dist_cont}.

The \texttt{[H|T]=Receivers} construct provides a method to firstly
extract the head element and the tail list, and secondly provide a
single variable bound to the whole list. This way could operate on the
whole list when sending the stop messages to the receivers, and on the
separate head element when sending the processing request to the next
receiver.

On \autoref{line:dist_cont} when append the list \texttt{[H]}
(containing only the single element \texttt{H}) to the back of the
list \texttt{T} in an efficient way. Thus we move the receiver that
have just been sent a message, to the back of the queue!


\subsubsection{RoundRobin Receiver}
\label{sec:roundrobin-receiver}

Just looking at the code below, the reader should notice it is nothing
more than a switching statements, but encapsulated based on the
messages received. But let us take a quick line by line explanation to
clear any doubts.

\begin{lstlisting}[name=hvp2]
hv_rr_rec(Name,Res) -> (*@\label{line:rrr-init}@*)
	receive 
		{stop} -> false; %Actually *any* terminating statement would suffice(*@\label{line:exit1}@*)
		{process,[rd,E,I]} -> Res!{rd,E,I,mds:reduce(mds:deriv(E,I))},hv_rr_rec(Name,Res);(*@\label{line:proc-rd}@*)
		{process,[null,E]} -> Res!{null,E,mds:null(E)},hv_rr_rec(Name,Res);(*@\label{line:proc-nul}@*)
		Other -> io:write(Other),throw(Other)(*@\label{line:other}@*)
		after 3000 ->	io:format("Timeout ~p quitting",[Name]),io:nl()(*@\label{line:timeout}@*)
	end.
\end{lstlisting}

\begin{description}
\item[\autoref{line:rrr-init}] The only parameter in there needed, is
  the \texttt{Res} variable, that tells the receiver where to send the
  results it is computing. The \texttt{Name} variable was used in
  debugging, and is referred to only in the timeout condition on \autoref{line:timeout}
\item[\autoref{line:exit1}] This \texttt{\{stop\}} message in
  \autoref{line:exit1}, could execute any terminating command, and the
  receiver would exit. We just choose it to return as \texttt{false},
  but true would have worked just as well.  
\item[\autoref{line:proc-rd} and \ref{line:proc-nul}] The crux and
  reason for this thread's existence, are theses two lines. All they
  do, is just call the relevant function(s) from the \texttt{mds}
  module, sent the result on to \texttt{Res}ult-receiver, a.k.a. the
  Reducer, and then loop back to themselves
\item[\autoref{line:other}] although not technically needed, nor
  useful in the production case, it was a good way to find typing
  errors in the core-loop. In essence the \texttt{Other} variable gets
  bound to \emph{any} message not matched by \autoref{line:exit1} to
  \autoref{line:proc-nul}. Then this get outputted to the console by
  \texttt{io:write/1} in a format that Erlang decided the message was,
  ie. list or tuple. As encore (to make sure things are noticed) it
  also throw an exception with \texttt{throw/1}
\item[\autoref{line:timeout}] We made an assumption here that the
  receiver should never have to wait more than
  3seconds\footnote{though this is something that should be
    parameterized in production code} before considering a problem
  with the distributor or other functions. At this point the name
  given to the receiver is useful in tracking which receiver have been
  stuck with which request.
\end{description}

\paragraph{Erlang's claims of ease of concurrency}
\label{sec:erlangs-claims-ease}

At this point in the code analysis, the author would have to emphasize
his appreciation of the cleanliness and ease of the concurrency
implementation, given the author's experience with TCP/IP on
MS-DOS\copyright, SUN-RPC remote procedure calls, Unix pipes and
messages. Given Erlang's ease of coding these concurrency features,
the author would, at present, concede only to a Smalltalk client
server model being easier to implement!

\subsubsection{AsAvailable Mappers}

As mentioned earlier, the round robin algorithm have the
problem that in its simplicity, it is not optimal to balance the load
across receivers, thus the reason for the AsAvailable algorithm. 

In this section, we will only highlight that which make the
AsAvailable different from the RoundRobin mapper. 

\paragraph{Spawning the AsAvailable receivers}

Just to be different, this time (especially as it is only an
initialization function) chose a bit different route to spawn the
receivers, especially as there was no need to keep track of them like
the RoundRobin distributor had to. In that regard I recursed the
\texttt{hv\_dist\_avail\_start/3} rather than to have a separate
function (like the RoundRobin case) as it was easier to do the
\texttt{N-1} on \autoref{line:das_recurse} than to write a separate
function for it.

 On reviewing the code, the author noticed 
\autoref{line:self_bad} being placed in a position where it will be
superfluously be called for each receiver, instead of once and the
result cached for each invocation. This being a result of the choice
made above.


\begin{lstlisting}[name=hvp2]
%Start the receivers and the distributor
hv_dist_avail_start(Timeout,_Res,0)	->
	hv_dist_available(Timeout,[]);
hv_dist_avail_start(Timeout,Res,N) when N>0 ->(*@\label{line:das_guard}@*)
	Dist=self(),(*@\label{line:self_bad}@*)
	spawn(fun() -> hv_rec_available(Timeout,"Receiver "++erlang:integer_to_list(N),Res,Dist) end),(*@\label{line:aa_rec_params}@*)
	hv_dist_avail_start(Timeout,Res,N-1).(*@\label{line:das_recurse}@*)
\end{lstlisting}

The main point to note here, compared to the round robin approach, is
on \autoref{line:aa_rec_params} where the receivers are passed the
result aggregator and the distributor's PIDs.

It would be noticed that \autoref{line:das_guard} have a guard
statement in it, as that would prevent fractions and negative numbers
to recurse to infinity and would  throw an error in that case
(no matching function). There are another case this would silently
fail at this point (but
timeout later) is when the call is made initially with 0 for the
number of threads.

\paragraph{AsAvailable distributor}

When the AsAvailable distributor starts up, it starts with a zero
knowledge of any receivers, thus its list of receivers are empty. In
this state \autoref{line:aad_start} matches, and we wait for an
\texttt{\{available,PID\}} message. Once we receive such a message, we
recurse, just this time with a non-empty list in the second parameter,
thus \autoref{line:aad_normal} matches.

\begin{lstlisting}[name=hvp2]
%The Available distributor
%First the "empty" case
hv_dist_available(Timeout,[]) ->(*@\label{line:aad_start}@*)
	receive
		{available,PID}-> hv_dist_available(Timeout,[PID])(*@\label{line:1_avail}@*)
	after Timeout -> 
		io:format("timeout distributor from waiting state~n")
	end;
\end{lstlisting}

In the case that we have a non-empty list of available receivers, the
function look very similar to the round robin case, except for
\autoref{line:add_avail} and \autoref{line:add_recurse}.

\begin{lstlisting}[name=hvp2]
%Receiver list not empty, "normal" case:
hv_dist_available(Timeout,[H|Tail]=Receivers)-> (*@\label{line:aad_normal}@*)
	receive
		{available,PID}->hv_dist_available(Timeout,[PID|Receivers]);(*@\label{line:add_avail}@*)
		{process,Param}->H!{process,Param},
						 hv_dist_available(Timeout,Tail);(*@\label{line:aad_recurse}@*)
		{stop}-> lists:foreach(fun(X)->X!{stop} end,Receivers);
		Other -> throw(Other)
		after Timeout ->
			io:format("Timeout distributor fron available state~n")
	end.
\end{lstlisting}

\begin{description}
\item[\autoref{line:add_avail}] this extra message is what we will
  receiver from a receiver that is idle and waiting for work. For the sake
  of simplicity, we prepend this receiver's PID to the front of the
  queue, thus make use of a LIFO\footnote{Last In, First Out} queue strategy.
\item[\autoref{line:aad_recurse}] This line differs from the
  RoundRobin case \ref{line:dist_cont} on page
  \pageref{line:dist_cont}, in so far as it discards the \texttt{Head}
  (whom we have just sent a process request, thus it is not available
  any more) and recurse with the \texttt{Tail} portion. If the
  \texttt{Tail} is an empty list, we should go back to
  \autoref{line:aad_start}, else \autoref{line:aad_normal}.
\end{description}

\paragraph{AsAvailable receiver}

The main difference between this and the round robin receiver, is the
dual receive blocks, and the reason for those two nearly identical
code blocks, is  that we have to tell the distributor we are available
for processing data, but not only that, but that we need to take a
couple of protocol resilience issues into consideration

\begin{lstlisting}[name=hvp2]
hv_rec_available(Timeout,Name,Res,Dist) ->
	%First we handle all stop/process messages on the queue
	receive
		{stop} ->	exit(0); %Need to do the exit here else continue to next receive ;((*@\label{line:must_exit}@*)
		{process,[rd,E1,I1]}-> 
			Res!{rd,E1,I1,mds:reduce(mds:deriv(E1,I1))},
			hv_rec_available(Timeout,Name,Res,Dist);
		{process,[null,E1]} ->
			Res!{null,E1,mds:null(E1)},
			hv_rec_available(Timeout,Name,Res,Dist);
		Other1 -> throw(Other1)(*@\label{line:imute1}@*)%Other1 here, as we already have Other below
		after 0 -> Dist!{available,self()} %Nothing in queue, so we let the Distributor know we are available(*@\label{line:rec_avail_again}@*)
	end,
	%Queue Empty when we get here, so lets wait :)
	receive
		{stop} ->	true;(*@\label{line:exit_true}@*)
		{process,[rd,E,I]} ->
			Res!{rd,E,I,mds:reduce(mds:deriv(E,I))},
			hv_rec_available(Timeout,Name,Res,Dist);
      		{process,[null,E]} ->
			Res!{null,E,mds:null(E)},
			hv_rec_available(Timeout,Name,Res,Dist);
		Other -> throw(Other)(*@\label{line:imute}@*)
		after Timeout ->
			io:format("Timeout ~p quiting ~n",[Name])(*@\label{line:aa_rec_timeout}@*)
	end.
\end{lstlisting}
\begin{description}
  \item[\autoref{line:must_exit} and \ref{line:exit_true}] The reason
    we can not exit in the first block with a simple \texttt{true}, is
    that it would then execute the second block, which is not what we
    want to do, so we forcibly \texttt{exit/1} this block
  \item[\autoref{line:imute1} and \ref{line:imute}] remember the
    discussion in \autoref{sec:imVar} where a variable inside a
    function's execution run, may not be bound to more than one value?
    This is a case where that is enforced by the compiler. It could be
    argued that it would never happen, but given the possibility that
    both blocks might get executed, the compiler enforces this rule to
    prevent the immutable variables to never be changed.
  \item[\autoref{line:rec_avail_again} and \ref{line:aa_rec_timeout}]
    these are the single reason why there is a need for two separate
    blocks, being that \autoref{line:rec_avail_again} will tell the
    distributor it is available only when the queue is empty (the
    \texttt{after 0 ->} timeout), and then the receiver will wait
    (block) for work from the distributor, with a ``normal'' timeout
    just like the round robin receiver.
\end{description}

This then closes the major algorithmic code discussions, and only
leaves us with the ancillary functions mentioned.


\section{Ancillary functions}
\label{sec:ancillary-functions}

\subsection{Adding to Work in Progress}
\label{sec:adding-work-progress}

The Work in Progress (WiP) is a list of the type of processing that
have been sent out to the distributors, and that we still have not
received any responses back for. It is a (in our implementation) a
simple list, but is perhaps a place to optimize if it grows too
big. This is especially as we prepend the new work to the beginning of
the list, but the expected responses to be received next are those at
the back of the list that would need to be removed, so in retrospect a
FIFO type queue would be a preferred implementation when optimizing.

\begin{lstlisting}[name=hvp2]
%For all the Sigma add {E,i} to the Work In Progress
add_wip(WiP, Type, E, [H]) -> [{Type, E, H}| WiP];(*@\label{line:AddHead}@*)
add_wip(WiP, Type, E, [H| SigmaT]) -> add_wip([{Type, E, H}| WiP], Type, E, SigmaT).(*@\label{line:AddList}@*)
\end{lstlisting}

Again we saw the use of specialization by the use of parameter
matching. Here we first check on line~\autoref{line:AddHead} if the fourth
parameter is a list with a single element (and we bound the variable
\texttt{H} to that single element), else we match on
line~\autoref{line:AddList} for a list with more than one element and we bind
the head (first element) to \texttt{H} while the tail (rest of the list
excluding the head) is bound to \texttt{SigmaT}.

\subsubsection{Mailbox debugging}

\begin{lstlisting}[name=hvp2]
 output_mailbox(N) ->
	receive
		Mess -> io:format("Message ~p~n:",[N]),
								io:write(Mess), io:nl(),
								output_mailbox(N+1)
	after 0 -> exit(123)
	end.
\end{lstlisting}

In the early stages of the AsAvailable mappers, we missed the problem
related to \autoref{line:rec_avail_again} and
\ref{line:aa_rec_timeout} on \pageref{line:rec_avail_again}. This
little function called as: \texttt{output\_mailbox(0)} was a lifesaver.


%%% Local Variables: 
%%% mode: latex
%%% TeX-master: "SPE780-project-dissertation"
%%% End: 
